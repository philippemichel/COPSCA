% Options for packages loaded elsewhere
\PassOptionsToPackage{unicode}{hyperref}
\PassOptionsToPackage{hyphens}{url}
\PassOptionsToPackage{dvipsnames,svgnames,x11names}{xcolor}
%
\documentclass[
  a4paperpaper,
  french,
  oneside,
  open=any]{scrreprt}

\usepackage{amsmath,amssymb}
\usepackage{lmodern}
\usepackage{iftex}
\ifPDFTeX
  \usepackage[T1]{fontenc}
  \usepackage[utf8]{inputenc}
  \usepackage{textcomp} % provide euro and other symbols
\else % if luatex or xetex
  \usepackage{unicode-math}
  \defaultfontfeatures{Scale=MatchLowercase}
  \defaultfontfeatures[\rmfamily]{Ligatures=TeX,Scale=1}
  \setmainfont[Numbers=OldStyle,Ligatures=TeX]{Faune}
  \setsansfont[Ligatures=TeX]{Myriad Pro}
\fi
% Use upquote if available, for straight quotes in verbatim environments
\IfFileExists{upquote.sty}{\usepackage{upquote}}{}
\IfFileExists{microtype.sty}{% use microtype if available
  \usepackage[]{microtype}
  \UseMicrotypeSet[protrusion]{basicmath} % disable protrusion for tt fonts
}{}
\makeatletter
\@ifundefined{KOMAClassName}{% if non-KOMA class
  \IfFileExists{parskip.sty}{%
    \usepackage{parskip}
  }{% else
    \setlength{\parindent}{0pt}
    \setlength{\parskip}{6pt plus 2pt minus 1pt}}
}{% if KOMA class
  \KOMAoptions{parskip=half}}
\makeatother
\usepackage{xcolor}
\setlength{\emergencystretch}{3em} % prevent overfull lines
\setcounter{secnumdepth}{5}
% Make \paragraph and \subparagraph free-standing
\ifx\paragraph\undefined\else
  \let\oldparagraph\paragraph
  \renewcommand{\paragraph}[1]{\oldparagraph{#1}\mbox{}}
\fi
\ifx\subparagraph\undefined\else
  \let\oldsubparagraph\subparagraph
  \renewcommand{\subparagraph}[1]{\oldsubparagraph{#1}\mbox{}}
\fi

\providecommand{\tightlist}{%
  \setlength{\itemsep}{0pt}\setlength{\parskip}{0pt}}\usepackage{longtable,booktabs,array}
\usepackage{calc} % for calculating minipage widths
% Correct order of tables after \paragraph or \subparagraph
\usepackage{etoolbox}
\makeatletter
\patchcmd\longtable{\par}{\if@noskipsec\mbox{}\fi\par}{}{}
\makeatother
% Allow footnotes in longtable head/foot
\IfFileExists{footnotehyper.sty}{\usepackage{footnotehyper}}{\usepackage{footnote}}
\makesavenoteenv{longtable}
\usepackage{graphicx}
\makeatletter
\def\maxwidth{\ifdim\Gin@nat@width>\linewidth\linewidth\else\Gin@nat@width\fi}
\def\maxheight{\ifdim\Gin@nat@height>\textheight\textheight\else\Gin@nat@height\fi}
\makeatother
% Scale images if necessary, so that they will not overflow the page
% margins by default, and it is still possible to overwrite the defaults
% using explicit options in \includegraphics[width, height, ...]{}
\setkeys{Gin}{width=\maxwidth,height=\maxheight,keepaspectratio}
% Set default figure placement to htbp
\makeatletter
\def\fps@figure{htbp}
\makeatother

\usepackage{booktabs}
\usepackage{longtable}
\usepackage{array}
\usepackage{multirow}
\usepackage{wrapfig}
\usepackage{float}
\usepackage{colortbl}
\usepackage{pdflscape}
\usepackage{tabu}
\usepackage{threeparttable}
\usepackage{threeparttablex}
\usepackage[normalem]{ulem}
\usepackage{makecell}
\usepackage{xcolor}
\definecolor{novo}{HTML}{27484b}
\makeatletter
\makeatother
\makeatletter
\makeatother
\makeatletter
\@ifpackageloaded{caption}{}{\usepackage{caption}}
\AtBeginDocument{%
\ifdefined\contentsname
  \renewcommand*\contentsname{Table des matières}
\else
  \newcommand\contentsname{Table des matières}
\fi
\ifdefined\listfigurename
  \renewcommand*\listfigurename{Liste des Figures}
\else
  \newcommand\listfigurename{Liste des Figures}
\fi
\ifdefined\listtablename
  \renewcommand*\listtablename{Liste des Tables}
\else
  \newcommand\listtablename{Liste des Tables}
\fi
\ifdefined\figurename
  \renewcommand*\figurename{Figure}
\else
  \newcommand\figurename{Figure}
\fi
\ifdefined\tablename
  \renewcommand*\tablename{Table}
\else
  \newcommand\tablename{Table}
\fi
}
\@ifpackageloaded{float}{}{\usepackage{float}}
\floatstyle{ruled}
\@ifundefined{c@chapter}{\newfloat{codelisting}{h}{lop}}{\newfloat{codelisting}{h}{lop}[chapter]}
\floatname{codelisting}{Listing}
\newcommand*\listoflistings{\listof{codelisting}{Liste des Listings}}
\makeatother
\makeatletter
\@ifpackageloaded{caption}{}{\usepackage{caption}}
\@ifpackageloaded{subcaption}{}{\usepackage{subcaption}}
\makeatother
\makeatletter
\@ifpackageloaded{tcolorbox}{}{\usepackage[many]{tcolorbox}}
\makeatother
\makeatletter
\@ifundefined{shadecolor}{\definecolor{shadecolor}{rgb}{.97, .97, .97}}
\makeatother
\makeatletter
\makeatother

\usepackage{hyphenat}
\usepackage{ifthen}
\usepackage{calc}
\usepackage{calculator}

\usepackage{graphicx}
\usepackage{wallpaper}

\usepackage{geometry}

\usepackage{graphicx}
\usepackage{geometry}
\usepackage{afterpage}
\usepackage{tikz}
\usetikzlibrary{calc}
\usetikzlibrary{fadings}
\usepackage[pagecolor=none]{pagecolor}


% Set the titlepage font families







% Set the coverpage font families

\ifLuaTeX
\usepackage[bidi=basic]{babel}
\else
\usepackage[bidi=default]{babel}
\fi
\babelprovide[main,import]{french}
% get rid of language-specific shorthands (see #6817):
\let\LanguageShortHands\languageshorthands
\def\languageshorthands#1{}
\ifLuaTeX
  \usepackage{selnolig}  % disable illegal ligatures
\fi
\usepackage[]{biblatex}
\addbibresource{stat.bib}
\IfFileExists{bookmark.sty}{\usepackage{bookmark}}{\usepackage{hyperref}}
\IfFileExists{xurl.sty}{\usepackage{xurl}}{} % add URL line breaks if available
\urlstyle{same} % disable monospaced font for URLs
\hypersetup{
  pdftitle={CopSCA},
  pdfauthor={Dr Philippe MICHEL},
  pdflang={fr-FR},
  colorlinks=true,
  linkcolor={blue},
  filecolor={Maroon},
  citecolor={Blue},
  urlcolor={Blue},
  pdfcreator={LaTeX via pandoc}}

\title{CopSCA}
\usepackage{etoolbox}
\makeatletter
\providecommand{\subtitle}[1]{% add subtitle to \maketitle
  \apptocmd{\@title}{\par {\large #1 \par}}{}{}
}
\makeatother
\subtitle{Plan d'analyse statistique V1.1}
\author{Dr Philippe MICHEL}
\date{27/04/2023}

\begin{document}
%%%%% begin titlepage extension code


\begin{titlepage}

%%% TITLE PAGE START

% Set up alignment commands
%Page
\newcommand{\titlepagepagealign}{
\ifthenelse{\equal{left}{right}}{\raggedleft}{}
\ifthenelse{\equal{left}{center}}{\centering}{}
\ifthenelse{\equal{left}{left}}{\raggedright}{}
}


\newcommand{\titleandsubtitle}{
% Title and subtitle
{\textcolor{novo}{\Huge{\bfseries{\nohyphens{CopSCA}}}}\par
}%

\vspace{\betweentitlesubtitle}
{
\textcolor{novo}{\huge{\nohyphens{Plan d'analyse statistique V1.1}}}\par
}}
\newcommand{\titlepagetitleblock}{
\titleandsubtitle
}

\newcommand{\authorstyle}[1]{{\large{#1}}}

\newcommand{\affiliationstyle}[1]{{\large{#1}}}

\newcommand{\titlepageauthorblock}{
{\authorstyle{\nohyphens{Dr Philippe MICHEL}{\textsuperscript{1}}}}}

\newcommand{\titlepageaffiliationblock}{
\hangindent=1em
\hangafter=1
{\affiliationstyle{
{1}.~Hôpital NOVO,~Unité de Soutien à la Recherche Clinique


\vspace{1\baselineskip} 
}}
}
\newcommand{\headerstyled}{%
{}
}
\newcommand{\footerstyled}{%
{\large{- \textbf{Marion DUPAIN} -- SAMU 95 -- Hôpital NOVO (Site
Pontoise)\newline - \textbf{Dr Olivier ANCELLI} -- SAMU 95 -- Hôpital
NOVO (Site Pontoise)\newline \newline Place du dosage
copeptine/troponine dans le diagnostic d'élimination des SCA non ST+
dans la prise en charge des douleur thoraciques non traumatiques en pré
et intra hospitalier chez l'adulte. \newline \raggedleft{\today}}}
}
\newcommand{\datestyled}{%
{27/04/2023}
}


\newcommand{\titlepageheaderblock}{\headerstyled}

\newcommand{\titlepagefooterblock}{
\footerstyled
}

\newcommand{\titlepagedateblock}{
\datestyled
}

%set up blocks so user can specify order
\newcommand{\titleblock}{\newlength{\betweentitlesubtitle}
\setlength{\betweentitlesubtitle}{\baselineskip}
{

{\titlepagetitleblock}
}

\vspace{4\baselineskip}
}

\newcommand{\authorblock}{{\titlepageauthorblock}

\vspace{2\baselineskip}
}

\newcommand{\affiliationblock}{{\titlepageaffiliationblock}

\vspace{1pt}
}

\newcommand{\logoblock}{}

\newcommand{\footerblock}{{\titlepagefooterblock}

\vspace{1pt}
}

\newcommand{\dateblock}{{\titlepagedateblock}

\vspace{0pt}
}

\newcommand{\headerblock}{}
\newgeometry{top=3in,bottom=1in,right=1in,left=1in}
% background image
\newlength{\bgimagesize}
\setlength{\bgimagesize}{0.5\paperwidth}
\LENGTHDIVIDE{\bgimagesize}{\paperwidth}{\theRatio} % from calculator pkg
\ThisULCornerWallPaper{\theRatio}{novo\_usrc.png}

\thispagestyle{empty} % no page numbers on titlepages


\newcommand{\vrulecode}{\rule{\vrulewidth}{\textheight}}
\newlength{\vrulewidth}
\setlength{\vrulewidth}{0.1cm}
\newlength{\B}
\setlength{\B}{\ifdim\vrulewidth > 0pt 0.05\textwidth\else 0pt\fi}
\newlength{\minipagewidth}
\ifthenelse{\equal{left}{left} \OR \equal{left}{right} }
{% True case
\setlength{\minipagewidth}{\textwidth - \vrulewidth - \B - 0.1\textwidth}
}{
\setlength{\minipagewidth}{\textwidth - 2\vrulewidth - 2\B - 0.1\textwidth}
}
\ifthenelse{\equal{left}{left} \OR \equal{left}{leftright}}
{% True case
\raggedleft % needed for the minipage to work
\vrulecode
\hspace{\B}
}{%
\raggedright % else it is right only and width is not 0
}
% [position of box][box height][inner position]{width}
% [s] means stretch out vertically; assuming there is a vfill
\begin{minipage}[b][\textheight][s]{\minipagewidth}
\titlepagepagealign
\titleblock

\authorblock

\affiliationblock

\vfill

\logoblock

\footerblock
\par

\end{minipage}\ifthenelse{\equal{left}{right} \OR \equal{left}{leftright} }{
\hspace{\B}
\vrulecode}{}
\clearpage
\restoregeometry
%%% TITLE PAGE END
\end{titlepage}
\setcounter{page}{1}

%%%%% end titlepage extension code\ifdefined\Shaded\renewenvironment{Shaded}{\begin{tcolorbox}[breakable, boxrule=0pt, borderline west={3pt}{0pt}{shadecolor}, interior hidden, frame hidden, enhanced, sharp corners]}{\end{tcolorbox}}\fi

\renewcommand*\contentsname{Table des matières}
{
\hypersetup{linkcolor=}
\setcounter{tocdepth}{2}
\tableofcontents
}
\begin{center}\rule{0.5\linewidth}{0.5pt}\end{center}

\hypertarget{guxe9nuxe9ralituxe9s}{%
\chapter{Généralités}\label{guxe9nuxe9ralituxe9s}}

Le risque \(\alpha\) retenu sera de 0,05 \& la puissance de 0,8.

Sauf indication contraires pour les tests simples les variables
numériques seront comparées par un test de Student si possible (
effectifs suffisants, distribution normales (Test de Shapiro-Wilk),
égalité des variances) auquel cas un test non paramétrique de Wilcoxon
sera utilisé. Un test du \(\chi^2\) sera utilisé pour les variables
discrètes sous réserve d'un effectif suffisant. À défaut un test de
Fischer sera utilisé. Des graphiques seront réalisés pour les résultats
importants (package \texttt{ggplot2} \autocite{ggplot}).

\hypertarget{nombre-de-cas}{%
\chapter{nombre de cas}\label{nombre-de-cas}}

Ce qui nous intéresse est de diagnostiquer les non malades c'est donc la
spécificité qui est le marqueur intéressant. Le premier dosage de la
troponine (\textless{} 6 h) a une spécificité connue de 92 \% en
présence de signes cliniques évocateurs.

Il faudrait donc 80 cas malades (on tolère une marge d'erreur de 5 \%).

Au vu de la littérature, on estime le pourcentage de malades à 30 dans
notre échantillon ce qui monte le total à \textbf{266 sujets}
nécessaires.

\hypertarget{donnuxe9es-manquantes}{%
\chapter{Données manquantes}\label{donnuxe9es-manquantes}}

Le décompte des données manquantes sera réalisé \& présenté par un
tableau ou un graphique. Les variables comportant trop de données
manquantes ou non utilisables ne seront pas prises en compte après
validation par le promoteur.

Après ce premier tri une imputation des données manquantes (package
\texttt{missMDA} \autocite{miss}) sera réalisée uniquement pour
l'analyse factorielle.

\hypertarget{description-de-la-population}{%
\chapter{Description de la
population}\label{description-de-la-population}}

\hypertarget{analyse-simple}{%
\section{Analyse simple}\label{analyse-simple}}

La description de la population concerne :

Un tableau présentera les valeurs démographiques \& clinique pour
l'ensemble de la population é pour chaque groupe. Les valeurs numériques
seront présentées en médiane \& quartiles, les valeurs discrètes en \%
avec son intervalle de confiance à 95 \%. Une différence entre les
groupes sera recherchée, item par item afin de valider au mieux
l'équivalence entre les groupes.

Une recherche de corrélation (package \texttt{corr} \autocite{corr})
entre les variables sera réalisée. Si certaines variables se montreront
trop corrélées elles pourront être exclues de l'étude après accord du
promoteur.

\hypertarget{analyse-factorielle}{%
\section{Analyse factorielle}\label{analyse-factorielle}}

Si le nombre de cas recueillis le permet une analyse factorielle en MCA
(Analyse de correspondances multiples - package \texttt{FactoMineR}
\autocite{facto}) sera réalisée.

Cette analyse ne pourra être qu'après transformation des variables
numériques en catégories \& imputation des données manquantes ce qui
n'est possible que si ces dernières ne sont pas trop nombreuses.

\hypertarget{objectif-principal}{%
\chapter{Objectif principal}\label{objectif-principal}}

\emph{Le couple troponine copeptine dosé lors du premier contact médical
permet-il d'exclure le diagnostic de SCA non ST + chez des patients
présentant une douleur thoracique de moins de 6 heures.}

Dans un premier temps on va chercher à définir un seuil diagnostic pour
la copeptine, le seuil de détection de la troponine étant considéré
comme connu. Ce seuil sera défini par l'analyse d'une courbe ROC.

Une fois ce seuil défini, on testera plusieurs couples
troponine/copeptine autour des seuils définis pour chercher le meilleur
compromis. On définira alors la sensibilité, la spécificité \& la valeur
prédictive négative pour plusieurs valeurs.

\hypertarget{objectifs-secondaires}{%
\chapter{Objectifs secondaires}\label{objectifs-secondaires}}

\hypertarget{objectif-1}{%
\section{Objectif 1}\label{objectif-1}}

\emph{Évaluation des performances de diagnostiques de l'association
troponine / copeptine selon les facteurs de risque cardiovasculaire
(âge, tabac, hypertension, antécédents ischémiques cardiaque, surpoids,
hérédité).}

Une courbe ROC sera tracée \& analysé dans chacun des sous-groupes pour
la copeptine. Si le résultat est très différent (plus de 20 \% d'écart)
par rapport à la population entière on cherchera par la même procédure
que pour le critère principal les meilleurs couples troponine/copeptine.

Il ne faut pas oublier que les effectifs des sous groupes seront très
faibles \& que les résultats obtenus n'auront qu'une valeur indicative.

\hypertarget{objectif-2}{%
\section{Objectif 2}\label{objectif-2}}

\emph{Évaluation de la durée du séjour du patient}

La durée de séjour des patient sera présentée en médiane avec avec les
quantiles pour les groupes positif ou négatif d'après le couple
troponine/copeptine.

\hypertarget{objectif-3}{%
\section{Objectif 3}\label{objectif-3}}

\emph{Évolution de l'association des dosages en fonction du délai entre
1ère douleur et prélèvements.}

Des corrélations simples seront réalisée : délai douleur/prélèvement \&
dosage de copeptine, délai douleur/prélèvement \& dosage de troponine.
Un test de corrélation non paramétrique de Spearman sera utilisé.

\hypertarget{technique}{%
\chapter{Technique}\label{technique}}

L'analyse statistique sera réalisée avec le logiciel
\textbf{R}\autocite{rstat} \& divers packages. Outre ceux cités dans le
texte ou utilisera en particulier \texttt{tidyverse} \autocite{tidy} \&
\texttt{baseph} \autocite{baseph}.

Un dépôt GitHub sera utilisé qui ne comprendra que le code \& non les
données ou résultats. Au besoin un faux tableau de données sera présenté
pour permettre des tests.

\url{https://github.com/philippemichel/COPSCA}


\printbibliography


\end{document}
